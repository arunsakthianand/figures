\documentclass{article}

\usepackage{tikz}
\usepackage{circuitikz}

\begin{document}

\begin{figure}[h!]
  \begin{center}
    \begin{circuitikz}
      \draw (0,0)
      to[V,v=$U_q$] (0,2) % The voltage source
      to[short] (2,2)
      to[R=$R_1$] (2,0) % The resistor
      to[short] (0,0);
      \draw (2,2)
      to[short] (4,2)
      to[L=$L_1$] (4,0)
      to[short] (2,0);
      \draw (4,2)
      to[short] (6,2)
      to[C=$C_1$] (6,0)
      to[short] (4,0);
   \end{circuitikz}
  \end{center}
\end{figure}

\begin{figure}[h!]
\begin{circuitikz}
  \draw (0,0) node[npn](npn1) {}
  (npn1.base) node[anchor=east] {B}
  (npn1.collector) node[anchor=south,xshift=0.5cm] {C}
  (npn1.emitter) node[anchor=north] {E};
  \draw (npn1.collector) to[R] ++(0,2);
\end{circuitikz}
\end{figure}

\begin{figure}[h!]
\begin{circuitikz}
  \draw (0,0) node[pnp](pnp1) {}
  (npn1.base) node[] {a}
  (npn1.collector) node[] {b}
  (npn1.emitter) node[] {c};
  \draw (npn1.collector) to[R=$R_c$] ++(0,2)
  to[short] (4,2.77);
  \draw (npn1.emitter) to[short] (0,-2)
  to[short] (4,-2)
  to[V,v=$V_EC$] (4,2.77);
  \draw (npn1.base) to[short] (-3,0)
  to[short] (-3,-0.5);
  signal
  \draw (-3,-1.5) to[short] (-3,-2)
  to[V,v=$V_BC$] (0,-2);
  \draw (npn1.emitter) to[short] (1.5,-0.77);
  
  \draw (1.5,-0.77) node(anchor=west){//};
  
\end{circuitikz}
\end{figure}


\end{document}